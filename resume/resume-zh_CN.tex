% !TEX TS-program = xelatex
% !TEX encoding = UTF-8 Unicode
% !Mode:: "TeX:UTF-8"

\documentclass{resume}
\usepackage{zh_CN-Adobefonts_external} % Simplified Chinese Support using external fonts (./fonts/zh_CN-Adobe/)
%\usepackage{zh_CN-Adobefonts_internal} % Simplified Chinese Support using system fonts
\usepackage{linespacing_fix} % disable extra space before next section
\usepackage{cite}

\begin{document}
\pagenumbering{gobble} % suppress displaying page number

\name{康宗}

\basicInfo{
  {(+86) 151-1632-6640} \textperiodcentered\ 
  {kangzong@stu.xjtu.edu.cn} \textperiodcentered\ 
  {https://github.com/JessonKang}}

\section{求职意向}
\begin{itemize}
  \item \textbf{本人情况} \quad 西安交通大学软件工程专业研二在读
  \item \textbf{意向岗位} \quad 后台开发工程师
  \item \textbf{入职时间} \quad 2020年7月可入职,可实习时间为2个月及以上
\end{itemize}


\section{\  教育背景}
\datedsubsection{\textbf{西安交通大学} \quad 陕西西安}{2018.09 -- 至今}
\textit{硕士}\qquad 软件工程
\datedsubsection{\textbf{湖南涉外经济学院} \quad 湖南长沙}{2014.09 -- 2018.06}
\textit{本科}\quad 计算机科学与技术

\section{\ 获奖经历/证书}
\datedsubsection{{西安交通大学一等奖学金}}{2019.10}
\datedsubsection{{西安交通大学二等奖学金}}{2018.10}
\datedsubsection{{国家励志奖学金}}{2016.10}
\datedsubsection{{国家励志奖学金}}{2015.10}
{英语证书 \qquad\qquad CET-6}

\section{专业技能}
% increase linespacing [parsep=0.5ex]
\begin{itemize}[parsep=0.2ex]
  \item \textbf{编程语言} \qquad 了解Java语言基础,了解一点C/C++和python语法
  \item \textbf{算法} \qquad\qquad 了解常用的数据结构和算法
  \item \textbf{操作系统} \qquad 了解操作系统原理,懂一些Linux的基本操作
  \item \textbf{网络} \qquad\qquad 了解TCP/UDP传输协议,以及http/https协议
    \item \textbf{数据库} \qquad\quad 了解数据库原理,懂一些MySQL、Redis原理
  
\end{itemize}

\section{\ 项目经历}
\datedsubsection{\textbf{接口鉴权系统}{\quad 个人项目}}{https://github.com/JessonKang/interface-authentication}
\qquad 该系统可以实现所有对调用服务端接口的请求进行鉴权,且只有被鉴权成功的请求才能成功调用到接口。该系统可以避免服务端被重放攻击,保证接口被调用时服务端的安全性。
\begin{itemize}
  \item 客户端将调用请求所涉及的URL和AppID、密码、时间戳拼接在一起,使用SHA-256算法加密生成一个token,将token及AppID随URL传至服务端;
  \item 服务端接收到请求后,根据AppID从数据库中取出密码,再结合相关信息用同样的加密算法生成新token,如果两个token相同则鉴权成功,反之鉴权失败;
  \item 生成token时加入时间戳(随机变量)这个参数可以优化token生成算法,防止产生哈希冲突,同时在服务端也可以借助时间戳设置失效时间窗口(如1分钟),因此提高了系统的性能和安全性。
\end{itemize}

\section{\ 学生工作/自我评价}
\datedsubsection{\textbf{班级团支书}}{2014.10-2018.06}
\begin{itemize}
    \item 负责组织和管理班级相关的工作
\end{itemize}
\textbf{自我评价}
\begin{itemize}
    \item 人生态度积极乐观、性格平和,沟通能力强,擅长团队合作
    \item 乐于钻研,喜欢挑战新事物
    \item 喜欢徒步、骑行等户外运动
\end{itemize}


% Reference Test
%\datedsubsection{\textbf{Paper Title\cite{zaharia2012resilient}}}{May. 2015}
%An xxx optimized for xxx\cite{verma2015large}
%\begin{itemize}
%  \item main contribution
%\end{itemize}


%% Reference
%\newpage
%\bibliographystyle{IEEETran}
%\bibliography{mycite}
\end{document}
